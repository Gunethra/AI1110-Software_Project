\let\negmedspace\undefined
\let\negthickspace\undefined
\documentclass[journal,12pt,twocolumn]{IEEEtran}
%\documentclass[conference]{IEEEtran}
%\IEEEoverridecommandlockouts
% The preceding line is only needed to identify funding in the first footnote. If that is unneeded, please comment it out.
\usepackage{svg}
\usepackage{tikz, pgfplots}
\usepackage{cite}
\usepackage{amsmath,amssymb,amsfonts,amsthm}
\usepackage{algorithmic}
\usepackage{graphicx,wrapfig}
\usepackage{textcomp}
\usepackage{xcolor}
\usepackage{txfonts}
\usepackage{listings}
\usepackage{enumitem}
\usepackage{mathtools}
\usepackage{gensymb}
\usepackage[breaklinks=true]{hyperref}
\usepackage{tkz-euclide} % loads  TikZ and tkz-base
\usepackage{listings}


\usetikzlibrary{positioning}
%
%\usepackage{setspace}
%\usepackage{gensymb}
%\doublespacing
%\singlespacing

%\usepackage{graphicx}
%\usepackage{amssymb}
%\usepackage{relsize}
%\usepackage[cmex10]{amsmath}
%\usepackage{amsthm}
%\interdisplaylinepenalty=2500
%\savesymbol{iint}
%\usepackage{txfonts}
%\restoresymbol{TXF}{iint}
%\usepackage{wasysym}
%\usepackage{amsthm}
%\usepackage{iithtlc}
%\usepackage{mathrsfs}
%\usepackage{txfonts}
%\usepackage{stfloats}
%\usepackage{bm}
%\usepackage{cite}
%\usepackage{cases}
%\usepackage{subfig}
%\usepackage{xtab}
%\usepackage{longtable} 
%\usepackage{multirow}
%\usepackage{algorithm}
%\usepackage{algpseudocode}
%\usepackage{enumitem}
%\usepackage{mathtools}
%\usepackage{tikz}
%\usepackage{circuitikz}
%\usepackage{verbatim}
%\usepackage{tfrupee}
%\usepackage{stmaryrd}
%\usetkzobj{all}
%    \usepackage{color}                                            %%
%    \usepackage{array}                                            %%
%    \usepackage{longtable}                                        %%
%    \usepackage{calc}                                             %%
%    \usepackage{multirow}                                         %%
%    \usepackage{hhline}                                           %%
%    \usepackage{ifthen}                                           %%
  %optionally (for landscape tables embedded in another document): %%
%    \usepackage{lscape}     
%\usepackage{multicol}
%\usepackage{chngcntr}
%\usepackage{enumerate}

%\usepackage{wasysym}
%\newcounter{MYtempeqncnt}
\DeclareMathOperator*{\Res}{Res}
%\renewcommand{\baselinestretch}{2}
\renewcommand\thesection{\arabic{section}}
\renewcommand\thesubsection{\thesection.\arabic{subsection}}
\renewcommand\thesubsubsection{\thesubsection.\arabic{subsubsection}}

\renewcommand\thesectiondis{\arabic{section}}
\renewcommand\thesubsectiondis{\thesectiondis.\arabic{subsection}}
\renewcommand\thesubsubsectiondis{\thesubsectiondis.\arabic{subsubsection}}

% correct bad hyphenation here
\hyphenation{op-tical net-works semi-conduc-tor}
\def\inputGnumericTable{}                                 %%


\definecolor{codegreen}{rgb}{0,0.6,0}
\definecolor{codegray}{rgb}{0.5,0.5,0.5}
\definecolor{codepurple}{rgb}{0.58,0,0.82}
\definecolor{backcolour}{rgb}{0.95,0.95,0.92}

\lstdefinestyle{python}{
    backgroundcolor=\color{backcolour},   
    commentstyle=\color{codegreen},
    keywordstyle=\color{magenta},
    numberstyle=\tiny\color{codegray},
    stringstyle=\color{codepurple},
    basicstyle=\ttfamily\footnotesize,
    breakatwhitespace=false,         
    breaklines=true,                 
    captionpos=b,                    
    keepspaces=true,                 
    numbers=left,                    
    numbersep=5pt,                  
    showspaces=false,                
    showstringspaces=false,
    showtabs=false,                  
    tabsize=2
}

\lstset{style = python}
%\lstset{
%language=tex,
%frame=single, 
%breaklines=true
%}


\begin{document}
%


\newtheorem{theorem}{Theorem}[section]
\newtheorem{problem}{Problem}
\newtheorem{proposition}{Proposition}[section]
\newtheorem{lemma}{Lemma}[section]
\newtheorem{corollary}[theorem]{Corollary}
\newtheorem{example}{Example}[section]
\newtheorem{definition}[problem]{Definition}
%\newtheorem{thm}{Theorem}[section] 
%\newtheorem{defn}[thm]{Definition}
%\newtheorem{algoithm}{Algorithm}[section]
%\newtheorem{cor}{Corollary}
\newcommand{\BEQA}{\begin{eqnarray}}
\newcommand{\EEQA}{\end{eqnarray}}
\newcommand{\define}{\stackrel{\triangle}{=}}
\newcommand\tab[1][1cm]{\hspace*{#1}}
\bibliographystyle{IEEEtran}
%\bibliographystyle{ieeetr}


\providecommand{\mbf}{\mathbf}
\providecommand{\pr}[1]{\ensuremath{\Pr\left(#1\right)}}
% Added from https://raw.githubusercontent.com/gadepall/digital-communication/main/probability/trans.tex %%
\newcommand*{\permcomb}[4][0mu]{{{}^{#3}\mkern#1#2_{#4}}}
\newcommand*{\perm}[1][-3mu]{\permcomb[#1]{P}}
\newcommand*{\comb}[1][-1mu]{\permcomb[#1]{C}}
\providecommand{\gauss}[2]{\mathcal{N}
\ensuremath{\left(#1,#2\right)}}
%%
\providecommand{\qfunc}[1]{\ensuremath{Q\left(#1\right)}}
\providecommand{\sbrak}[1]{\ensuremath{{}\left[#1\right]}}
\providecommand{\lsbrak}[1]{\ensuremath{{}\left[#1\right.}}
\providecommand{\rsbrak}[1]{\ensuremath{{}\left.#1\right]}}
\providecommand{\brak}[1]{\ensuremath{\left(#1\right)}}
\providecommand{\lbrak}[1]{\ensuremath{\left(#1\right.}}
\providecommand{\rbrak}[1]{\ensuremath{\left.#1\right)}}
\providecommand{\cbrak}[1]{\ensuremath{\left\{#1\right\}}}
\providecommand{\lcbrak}[1]{\ensuremath{\left\{#1\right.}}
\providecommand{\rcbrak}[1]{\ensuremath{\left.#1\right\}}}
\theoremstyle{remark}
\newtheorem{rem}{Remark}
\newcommand{\sgn}{\mathop{\mathrm{sgn}}}
\providecommand{\abs}[1]{\(left\)vert#1\(right\)vert}
\providecommand{\res}[1]{\Res\displaylimits_{#1}}
\providecommand{\norm}[1]{\(left\)lVert#1\(right\)rVert}
%\providecommand{\norm}[1]{\lVert#1\rVert}
\providecommand{\mtx}[1]{\mathbf{#1}}
\providecommand{\mean}[1]{E\(left\)[ #1 \(right\)]}
\providecommand{\fourier}{\overset{\mathcal{F}}{ \rightleftharpoons}}
%\providecommand{\hilbert}{\overset{\mathcal{H}}{ \rightleftharpoons}}
\providecommand{\system}{\overset{\mathcal{H}}{ \longleftrightarrow}}
%\newcommand{\solution}[2]{\textbf{Solution:}{#1}}
\newcommand{\solution}{\noindent \textbf{Solution: }}
\newcommand{\cosec}{\,\text{cosec}\,}
\providecommand{\dec}[2]{\ensuremath{\overset{#1}{\underset{#2}{\gtrless}}}}
\newcommand{\myvec}[1]{\ensuremath{\begin{pmatrix}#1\end{pmatrix}}}
\newcommand{\mydet}[1]{\ensuremath{\begin{vmatrix}#1\end{vmatrix}}}

\let\vec\mathbf

\vspace{3cm}

\title{
\textbf {Assignment 2}\\ \large \textbf{AI1110}: Probability and Random Variables\\Indian Institute of Techonology, Hyderabad
}
\author{Gunethra Bommineni$^{*}$% <-this % stops a space
	\thanks{*The student is with the Department
		of Electrical Engineering, Indian Institute of Technology, Hyderabad
		502285 India e-mail: ee22btech11205@iith.ac.in.}
  }

\maketitle

\newpage


\bigskip
\renewcommand{\thefigure}{\theenumi}
\renewcommand{\thetable}{\theenumi}
\textbf{Introduction\\}
In this project we were asked to create a music player which can shuffle songs so that no song is repeated until all songs have been played using Numpy and other libraries in Python. I have implemented it using the libraries os, numpy, and mixer from pygame. It features audio play and pause and audio skipping using a python script.
\begin{lstlisting}[language=Python, caption=Libraries used]
import os
import numpy as np
import pygame
from pygame import mixer
\end{lstlisting}

\textbf{Shuffling\\}
Numpy library was used to create a list whose size was measured by the `num\_songs` variable. The list of songs was given as input and converted  array `song` and another np array of same size was created using empty-like function in numpy. The new np array with the name `shuffled\_songs` would be used to store the shuffled order. The songs were indexed using `np.arange` and shuffled using `np.random.shuffle`.\\
Python function for shuffling:\cite{shuffle}
\begin{lstlisting}[language=Python, caption=Shuffling function]
def song_shuffler(songs):
    songs = np.array(songs)
    num_songs = len(songs)
    shuffled_songs = np.empty_like(songs)
    index_pool = np.arange(num_songs)

    np.random.shuffle(index_pool)

    for i in range(num_songs):
        shuffled_songs[i] = songs[index_pool[i]]
        if i < num_songs - 1:
            np.delete(index_pool, np.where(index_pool == index_pool[i]))

    return shuffled_songs
\end{lstlisting}

\textbf{Player\\}
The audio playback is taken care of, by mixer from the pygame library. The library is initialized by `pygame.mixer.init()`. The function `play\_songs` receives a parameter `songs`, which in this case, is a numpy array of shuffled song indices(shuffled\_songs). The variable `song\_path` is a string which stores the path of the required song. The path is a concatenation of the Folder path and the file name of the song. The `pygame .mixer.music.play()` command is responsible for playing the song. The while loop waits for user input to control the playback. If the user enters
\begin{itemize}
    \item pause; the playback is paused and `pause()` function is invoked
    \item resume; it returns "Song not paused!"
    \item next; it plays the next song in the list
    \item quit; it exits the player
\end{itemize}

Python function for playback:\cite{play}

\begin{lstlisting}[language=Python, caption=Play function]
def play_songs(songs):
    pygame.mixer.init()
    for song in songs:
        print("Now playing:", song)
        song_path = os.path.join(songs_folder, song)
        print("From:", song_path)
        pygame.mixer.music.load(song_path)
        pygame.mixer.music.play()
        while pygame.mixer.music.get_busy():
            print('*****************') #For cosmetic purpose
            command = input("Enter command (pause/resume/next/quit): ")

            if command.lower() == "pause":
                pygame.mixer.music.pause()
                pause()
            elif command.lower() == "resume":
                print("Song not paused!")
            elif command.lower() == "next":
                pygame.mixer.music.stop()
                break
            elif command.lower() == "quit":
                quit()
                return
\end{lstlisting}

When the `pause()` function is invoked, the it gives the user three options for playback. The available commands are "resume", "next", and "quit". If the user enters
\begin{itemize}
    \item resume; playback is resumed
    \item next; it plays the next song in the list
    \item quit; it exits the player
\end{itemize}
Python function for playback:\cite{pause}
\begin{lstlisting}[language=Python, caption=Pause function]
def pause():
    print("Paused")
    print('*****************') #For cosmetic purpose
    pause_command = input("Enter command (resume/next/quit): ")
    
    if pause_command.lower() == "resume":
        print("Resuming...")
        pygame.mixer.music.unpause()
    elif pause_command.lower() == "next":
        pygame.mixer.music.stop()
    elif pause_command.lower() == "quit":
        quit()
        return
\end{lstlisting}

\textbf{The main code\\}
The `songs\_folder` variable holds the path to the folder where the songs are located. The for loop initiates the player to play the songs in the shuffled order.
\begin{lstlisting}[language=Python, caption=Main code which invokes the functions]
songs = ['IMG_0553.wav', 'IMG_0555.wav', 'IMG_0556.wav', 'IMG_0557.wav', 'IMG_0558.wav', 'IMG_0559.wav', 'IMG_0560.wav', 'IMG_0561.wav', 'IMG_0562.wav', 'IMG_0563.wav', 'IMG_0565.wav', 'IMG_0566.wav', 'IMG_0567.wav', 'IMG_0568.wav', 'IMG_0569.wav', 'IMG_0570.wav', 'IMG_0571.wav', 'IMG_0572.wav', 'IMG_0574.wav', 'IMG_0575.wav']

shuffled_songs = song_shuffler(songs)

for song in shuffled_songs:
    print('*****************') #For cosmetic purpose
    play_songs(shuffled_songs)
\end{lstlisting}

\bibliography{bibliography}
\end{document}
